\documentclass[a4paper, 12pt]{article}

% Поддержка русского языка
\usepackage[T2A]{fontenc}
\usepackage[utf8]{inputenc}
\usepackage[english,russian]{babel}

% Остальные пакеты
\usepackage{amsmath}
\usepackage{graphicx}
\usepackage{hyperref}
\usepackage{listings}
\usepackage{xcolor}
\usepackage{tikz}

% Определение языка MC51
\lstdefinelanguage{mc51}{
    morekeywords={MOV, ADD, SUB, MUL, DIV, INC, DEC, DJNZ, CJNE, JC, JNC, JZ, JNZ, JB, JNB, JBC, ACALL, LCALL, RET, RETI, NOP, PUSH, POP, XCH, XCHD, SWAP, DA, CLR, SETB, CPL, ANL, ORL, XRL, RL, RLC, RR, RRC, JMP, SJMP, AJMP, LJMP, A, B, R0, R1, R2, R3, R4, R5, R6, R7, DPTR, PC, SP, PSW, ACC},
    sensitive=true,
    morecomment=[l]{;},
    morestring=[b]",
    morestring=[b]',
    basicstyle=\ttfamily\small,
    numbers=left,
    numberstyle=\tiny,
    frame=single,
    breaklines=true,
    postbreak=\mbox{\textcolor{red}{$\hookrightarrow$}\space},
    commentstyle=\color{green!60!black},
    stringstyle=\color{red},
    keywordstyle=\color{blue},
    showstringspaces=false,
    extendedchars=true,
    inputencoding=utf8
}

% Настройка гиперссылок
\hypersetup{
    unicode=true,
    colorlinks=true,
    linkcolor=blue,
    urlcolor=blue
}

\title{ABI Document for AT89S4051}
\author{Сухов А.А.}
\date{\today}

\begin{document}

\maketitle

\tableofcontents

\section{Введение}
Данный документ описывает соглашение о бинарном интерфейсе (Application Binary Interface, ABI) для микроконтроллера AT89S4051. 

\subsection{Цели документа}
\begin{itemize}
    \item Определить стандарт взаимодействия между компонентами программного обеспечения на уровне машинного кода
    \item Обеспечить совместимость между кодом, написанным на разных языках программирования
    \item Описать правила использования регистров и организации памяти
    \item Стандартизировать механизмы вызова функций и передачи параметров
\end{itemize}

\subsection{Область применения}
Документ предназначен для:
\begin{itemize}
    \item Разработчиков компиляторов
    \item Программистов, работающих с ассемблером
    \item Разработчиков системного программного обеспечения
    \item Специалистов по оптимизации кода
\end{itemize}

\section{Общее описание архитектуры C51}
\subsection{Основные характеристики AT89S4051}
AT89S4051 - это 8-битный микроконтроллер, основанный на архитектуре MCS-51 (8051), имеющий следующие характеристики:

\begin{itemize}
    \item Тактовая частота: до 24 МГц
    \item Напряжение питания: 4.0V - 5.5V
    \item Температурный диапазон: -40°C до +85°C
    \item Энергонезависимая память программ: 4KB Flash
    \item Оперативная память: 128 байт
    \item Количество циклов перепрограммирования: минимум 1000
    \item Время хранения данных: 10 лет при 85°C
\end{itemize}

\subsection{Архитектурные особенности}
\subsubsection{Процессорное ядро}
\begin{itemize}
    \item 8-битное АЛУ с поддержкой арифметических и логических операций
    \item Аппаратное умножение и деление
    \item Побитовая адресация
    \item Поддержка булевых операций
\end{itemize}

\subsubsection{Система команд}
\begin{itemize}
    \item Всего 111 инструкций
    \item Большинство инструкций выполняется за 1-2 машинных цикла
    \item Длина инструкций: 1-3 байта
    \item Типы адресации:
    \begin{itemize}
        \item Прямая
        \item Косвенная
        \item Регистровая
        \item Непосредственная
        \item Индексная
    \end{itemize}
\end{itemize}

\subsection{Регистры}
\subsubsection{Регистры общего назначения}
\begin{itemize}
    \item \textbf{R0-R7}: 8-битные регистры общего назначения
    \begin{itemize}
        \item Организованы в 4 банка по 8 регистров
        \item Выбор активного банка через биты RS0 и RS1 регистра PSW
        \item Могут использоваться для временного хранения данных
    \end{itemize}
    \item \textbf{ACC (A)}: Аккумулятор
    \begin{itemize}
        \item Основной регистр для арифметических и логических операций
        \item Источник/приемник для операций с памятью
        \item Используется для возврата 8-битных значений из функций
    \end{itemize}
    \item \textbf{B}: Вспомогательный регистр
    \begin{itemize}
        \item Второй операнд для операций умножения и деления
        \item Может использоваться как регистр общего назначения
    \end{itemize}
\end{itemize}

\subsubsection{Специальные регистры}
\begin{itemize}
    \item \textbf{DPTR}: 16-битный указатель данных
    \begin{itemize}
        \item Состоит из DPH (старший байт) и DPL (младший байт)
        \item Используется для адресации внешней памяти
        \item Применяется для табличных преобразований
    \end{itemize}
    \item \textbf{PC}: Счетчик команд
    \begin{itemize}
        \item 16-битный регистр
        \item Указывает на следующую исполняемую инструкцию
        \item Автоматически инкрементируется
    \end{itemize}
    \item \textbf{SP}: Указатель стека
    \begin{itemize}
        \item 8-битный регистр
        \item Указывает на последний занятый байт в стеке
        \item После сброса инициализируется значением 07h
    \end{itemize}
\end{itemize}

\section{Формат бинарных файлов}
\subsection{Формат Intel HEX}
Intel HEX формат представляет собой текстовый формат для представления бинарных данных. Каждая строка (запись) имеет следующий формат:

\begin{verbatim}
:LLAAAATT[DD...]CC
\end{verbatim}

где:
\begin{itemize}
    \item \textbf{:} - маркер начала записи
    \item \textbf{LL} - количество байт данных в записи (1 байт)
    \item \textbf{AAAA} - адрес начала данных (2 байта)
    \item \textbf{TT} - тип записи (1 байт)
    \item \textbf{DD} - данные (LL байт)
    \item \textbf{CC} - контрольная сумма (1 байт)
\end{itemize}

\subsection{Типы записей}
\begin{itemize}
    \item \textbf{00 - Data Record}
    \begin{itemize}
        \item Содержит данные для загрузки в память
        \item Адрес указывает место загрузки
        \item Наиболее часто используемый тип записи
    \end{itemize}
    \item \textbf{01 - End Of File Record}
    \begin{itemize}
        \item Обозначает конец файла
        \item Не содержит данных (LL = 00)
        \item Должна быть последней записью в файле
    \end{itemize}
    \item \textbf{02 - Extended Segment Address Record}
    \begin{itemize}
        \item Задает базовый адрес для последующих записей
        \item Содержит 16-битный сегментный адрес
        \item Используется для адресации более 64KB
    \end{itemize}
    \item \textbf{03 - Start Segment Address Record}
    \begin{itemize}
        \item Определяет начальное значение CS:IP
        \item Используется для x86 архитектуры
    \end{itemize}
    \item \textbf{04 - Extended Linear Address Record}
    \begin{itemize}
        \item Задает старшие 16 бит 32-битного адреса
        \item Используется для 32-битной адресации
    \end{itemize}
    \item \textbf{05 - Start Linear Address Record}
    \begin{itemize}
        \item Задает значение EIP
        \item Используется в 32-битных системах
    \end{itemize}
\end{itemize}

\subsection{Пример Intel HEX файла}
\begin{verbatim}
:10000000C3F000757A0A757B14C374147A007B0080D2
:10001000FE120100E5827A007B01F0227867786A22A9
:00000001FF
\end{verbatim}

\section{Соглашение о вызовах и передаче данных}
\subsection{Передача параметров}
\subsubsection{Через регистры}
Первые три параметра передаются через регистры:
\begin{itemize}
    \item \textbf{R7}: Первый параметр
    \item \textbf{R6}: Второй параметр
    \item \textbf{R5}: Третий параметр
\end{itemize}

\subsubsection{Через стек}
Параметры, начиная с четвертого, передаются через стек:
\begin{itemize}
    \item Параметры помещаются в стек в обратном порядке
    \item Каждый параметр занимает минимум 1 байт
    \item За освобождение стека отвечает вызывающая функция
\end{itemize}

\subsection{Возврат значений}
\subsubsection{8-битные значения}
\begin{itemize}
    \item Возвращаются в аккумуляторе (A)
    \item Флаги PSW отражают результат операции
\end{itemize}

\subsubsection{16-битные значения}
\begin{itemize}
    \item Возвращаются в DPTR
    \item DPH содержит старший байт
    \item DPL содержит младший байт
\end{itemize}

\subsubsection{32-битные значения}
\begin{itemize}
    \item Возвращаются через регистры R7:R6:R5:R4
    \item R7 содержит самый старший байт
    \item R4 содержит самый младший байт
\end{itemize}

\section{Память и ее организация}
\subsection{Карта памяти}
\subsubsection{Память программ}
\begin{itemize}
    \item Диапазон: 0x0000-0x0FFF
    \item Размер: 4KB Flash
    \item Только для чтения во время выполнения
    \item Может быть запрограммирована через SPI
\end{itemize}

\subsubsection{Внутреннее ОЗУ}
\begin{itemize}
    \item Диапазон: 0x0000-0x007F
    \item Размер: 128 байт
    \item Организация:
    \begin{itemize}
        \item 0x00-0x1F: Банки регистров
        \item 0x20-0x2F: Битовая область
        \item 0x30-0x7F: Область общего назначения
    \end{itemize}
\end{itemize}

\subsubsection{Специальные регистры (SFR)}
\begin{itemize}
    \item Диапазон: 0x0080-0x00FF
    \item Включает управляющие регистры
    \item Регистры портов ввода/вывода
    \item Регистры таймеров и прерываний
\end{itemize}

\subsection{Организация стека}
\subsubsection{Характеристики стека}
\begin{itemize}
    \item Расположен во внутреннем ОЗУ
    \item Растет снизу вверх
    \item Максимальный размер ограничен размером ОЗУ
    \item SP указывает на последний занятый байт
\end{itemize}

\subsubsection{Операции со стеком}
\begin{itemize}
    \item \textbf{PUSH}:
    \begin{itemize}
        \item SP инкрементируется
        \item Данные записываются по адресу [SP]
    \end{itemize}
    \item \textbf{POP}:
    \begin{itemize}
        \item Данные читаются из адреса [SP]
        \item SP декрементируется
    \end{itemize}
    \item \textbf{CALL}:
    \begin{itemize}
        \item Адрес возврата (2 байта) помещается в стек
        \item SP увеличивается на 2
    \end{itemize}
\end{itemize}

\section{Примеры кода}
\subsection{Базовые операции}
\subsubsection{Сложение двух чисел}
\begin{lstlisting}[language={mc51}]
    MOV  A, #10      ; Загрузка первого числа
    MOV  R7, #20     ; Загрузка второго числа
    ADD  A, R7       ; Сложение
    MOV  R0, A       ; Сохранение результата
\end{lstlisting}

\subsubsection{Умножение}
\begin{lstlisting}[language={mc51}]
    MOV  A, #5       ; Первый множитель
    MOV  B, #3       ; Второй множитель
    MUL  AB          ; Умножение
    ; Результат в A (младший байт)
    ; и B (старший байт)
\end{lstlisting}

\subsection{Работа с функциями}
\subsubsection{Пример вызова функции}
\begin{lstlisting}[language={mc51}]
Main:
    MOV  R7, #10     ; Первый параметр
    MOV  R6, #20     ; Второй параметр
    CALL Sum         ; Вызов функции
    MOV  R5, A       ; Сохранение результата
    RET

Sum:
    MOV  A, R7       ; Загрузка первого параметра
    ADD  A, R6       ; Сложение с вторым параметром
    RET              ; Возврат результата в A
\end{lstlisting}

\subsubsection{Функция с локальными переменными}
\begin{lstlisting}[language={mc51}]
Function:
    ; Сохранение используемых регистров
    PUSH ACC
    PUSH B
    
    ; Работа с локальными данными
    MOV  R0, #0      ; Локальная переменная
    
    ; Восстановление регистров
    POP  B
    POP  ACC
    RET
\end{lstlisting}

\subsection{Работа со стеком}
\begin{lstlisting}[language={mc51}]
    ; Сохранение контекста
    PUSH ACC         ; Сохранение аккумулятора
    PUSH PSW         ; Сохранение флагов
    PUSH DPH         ; Сохранение DPTR
    PUSH DPL
    
    ; Код функции
    
    ; Восстановление контекста
    POP  DPL         ; Восстановление DPTR
    POP  DPH
    POP  PSW         ; Восстановление флагов
    POP  ACC         ; Восстановление аккумулятора
\end{lstlisting}

\section{Заключение}
\subsection{Основные положения}
Данный ABI документ определяет:
\begin{itemize}
    \item Стандартные соглашения для компиляции и компоновки программ
    \item Правила взаимодействия между модулями на уровне машинного кода
    \item Механизмы вызова функций и передачи параметров
    \item Организацию памяти и использование регистров
\end{itemize}

\subsection{Рекомендации по использованию}
\begin{itemize}
    \item Строго следовать соглашениям о вызовах
    \item Учитывать ограничения архитектуры
    \item Оптимизировать использование регистров
    \item Правильно организовывать работу со стеком
\end{itemize}

\subsection{Совместимость}
Соблюдение описанных соглашений обеспечивает:
\begin{itemize}
    \item Совместимость между различными компиляторами
    \item Возможность использования библиотек
    \item Надежность межмодульного взаимодействия
    \item Предсказуемое поведение программ
\end{itemize}

\end{document}